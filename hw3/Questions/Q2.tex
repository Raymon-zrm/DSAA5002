\section*{Q2 [20 Marks]}

Use the similarity matrix in Table 2 to perform single-link hierarchical clustering. \\
Show your results by drawing a dendrogram. The dendrogram should clearly show the order in which the clusters are merged.
\begin{table}[h]
    \centering
    \begin{tabular}{@{}|c|c|c|c|c|c|@{}}
        \hline
        & p1   & p2   & p3   & p4   & p5   \\
        \hline
        p1 & 1.00 & 0.10 & 0.41 & 0.55 & 0.35 \\
        \hline
        p2 & 0.10 & 1.00 & 0.64 & 0.47 & 0.98 \\
        \hline
        p3 & 0.41 & 0.64 & 1.00 & 0.44 & 0.85 \\
        \hline
        p4 & 0.55 & 0.47 & 0.44 & 1.00 & 0.76 \\
        \hline
        p5 & 0.35 & 0.98 & 0.85 & 0.76 & 1.00 \\
        \hline
    \end{tabular}
    \caption{Similarity matrix for Q2}
\end{table}

\subsection*{Solution:}

Regenerate the matrix by eliminating the diagonal and the upper triangle:

\begin{table}[h]
    \centering
    \begin{tabular}{@{}c|cccc@{}}
        \hline
        & p1 & p2 & p3 & p4 \\
        \hline

        p2 & 0.10 &      &      &     \\
        p3 & 0.41 & 0.64 &      &     \\
        p4 & 0.55 & 0.47 & 0.44 &     \\
        p5 & 0.35 & 0.98 & 0.85 & 0.76 \\
    \end{tabular}
\end{table}

According to the single-link algorithm, we merge the two points with the max similarity.
And the new similarity is the max similarity between the two points and the other points.
Then we find that the max similarity is 0.98 between p2 and p5.

\begin{table}[H]
    \centering
    \begin{tabular}{@{}c|ccc@{}}
        \hline
        & p1 & p2\&5 & p3 \\
        \hline

        p2\&5 & 0.35 &      &      \\
        p3 & 0.41 & 0.85 &      \\
        p4 & 0.55 & 0.76 & 0.44 \\
    \end{tabular}
\end{table}

In the new similarity matrix, we find that the max similarity is 0.85 between p3 and p2\&5.

\begin{table}[!h]
    \centering
    \begin{tabular}{@{}c|cc@{}}
        \hline
        & p1 & p2\&3\&5 \\
        \hline

        p2\&3\&5 & 0.41 &      \\
        p4 & 0.55 & 0.76 \\
    \end{tabular}
\end{table}

In the new similarity matrix, we find that the max similarity is 0.76 between p4 and p2\&3\&5.

\begin{table}[!h]
    \centering
    \begin{tabular}{@{}c|c@{}}
        \hline
        & p1 \\
        \hline

        p2\&3\&4\&5 & 0.55 \\
    \end{tabular}
\end{table}

Hence we get the cluster tree:

\begin{center}
\begin{tikzpicture}
    % 定义样式
    \tikzstyle{level} = [circle, fill, inner sep=1.5pt]
    \tikzstyle{label} = [text width=2cm, align=center]

    % 绘制横轴
    \draw (0,0) -- (6,0);
    \node[align=center, anchor=east] at (-0.5,0) {Level};
    \foreach \x in {0,1,2,3,4,5}
        \draw (\x,0) -- (\x,-0.1) node[below] {\x};

    % 绘制树状图
    \node[label] at (-0.2,-1) {p1};
    \node[label] at (-0.2,-2) {p4};
    \node[label] at (-0.2,-3) {p3};
    \node[label] at (-0.2,-4) {p2};
    \node[label] at (-0.2,-5) {p5};

    % 绘制连接线
    \draw (0,-4) -| (1,-4.5);
    \draw (0,-5) -| (1,-4.5);
    \draw (1,-4.5) -| (2,-3.75);
    \draw (0,-3) -| (2,-3.75);
    \draw (2,-3.75) -| (3,-2.875);
    \draw (0,-2) -| (3,-2.875);
    \draw (3,-2.875) -| (4,-1.9375);
    \draw (0,-1) -| (4,-1.9375);
    \draw (4,-1.9375) -| (5,-1.9375);

    % 绘制合并层级的点
    \node[level] at (1,-4.5) {}; % Level 1
    \node[level] at (2,-3.75) {}; % Level 2
    \node[level] at (3,-2.875) {}; % Level 3
    \node[level] at (4,-1.9375) {}; % Level 3
\end{tikzpicture}
\end{center}