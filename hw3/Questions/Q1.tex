\section*{Q1 [20 Marks]}

Apply the agglomerative hierarchical clustering algorithm with the following distance matrix and the single linkage. Plot the cluster tree and mark out all the merging levels.

\begin{table}[h]
    \centering
    \begin{tabular}{@{}c|cccc@{}}
        \hline
        & 1 & 2 & 3 & 4 \\
        \hline

        2 & 2.33 &      &      &     \\
        3 & 3.15 & 1.30 &      &     \\
        4 & 1.90 & 1.50 & 3.70 &     \\
        5 & 3.01 & 0.47 & 1.40 & 1.82 \\
    \end{tabular}
    \caption{distance matrix}
\end{table}

\subsection*{Solution:}

According to the single-link algorithm, we merge the two points with the smallest distance.
And the new distance is the smallest distance between the two points and the other points.
Then we find that the smallest distance is 0.47 between 2 and 5.
Merge them and we get new distance matrix:

\begin{table}[h]
    \centering
    \begin{tabular}{@{}c|ccc@{}}
        \hline
        & 1  & (2\&5) & 3 \\
        \hline

        (2\&5) & 2.33 &      &     \\
        3 & 3.15 & 1.30 &     \\
        4 & 1.90 & 1.50 & 3.70      \\
    \end{tabular}
\end{table}

Then the distance 1.30 between 3 and (2\&5) is the smallest one.

\begin{table}[h]
    \centering
    \begin{tabular}{@{}c|cc@{}}
        \hline
        & 1  & (2\&3\&5) \\
        \hline

        (2\&3\&5) & 2.33 &     \\
        4 & 1.90 & 1.50 \\
    \end{tabular}
\end{table}

Then the distance 1.50 between (2\&3\&5) and 4 is the smallest one.

\begin{table}[!h]
    \centering
    \begin{tabular}{@{}c|c@{}}
        \hline
        & 1  \\
        \hline

        (2\&3\&4\&5) & 1.90 \\
    \end{tabular}
\end{table}

Hence we get the cluster tree:

\begin{center}
\begin{tikzpicture}
    % 定义样式
    \tikzstyle{level} = [circle, fill, inner sep=1.5pt]
    \tikzstyle{label} = [text width=2cm, align=center]

    % 绘制横轴
    \draw (0,0) -- (6,0);
    \node[align=center, anchor=east] at (-0.5,0) {Level};
    \foreach \x in {0,1,2,3,4,5}
        \draw (\x,0) -- (\x,-0.1) node[below] {\x};

    % 绘制树状图
    \node[label] at (-0.1,-1) {1};
    \node[label] at (-0.1,-2) {4};
    \node[label] at (-0.1,-3) {3};
    \node[label] at (-0.1,-4) {2};
    \node[label] at (-0.1,-5) {5};

    % 绘制连接线
    \draw (0,-4) -| (1,-4.5);
    \draw (0,-5) -| (1,-4.5);
    \draw (1,-4.5) -| (2,-3.75);
    \draw (0,-3) -| (2,-3.75);
    \draw (2,-3.75) -| (3,-2.875);
    \draw (0,-2) -| (3,-2.875);
    \draw (3,-2.875) -| (4,-1.9375);
    \draw (0,-1) -| (4,-1.9375);
    \draw (4,-1.9375) -| (5,-1.9375);

    % 绘制合并层级的点
    \node[level] at (1,-4.5) {}; % Level 1
    \node[level] at (2,-3.75) {}; % Level 2
    \node[level] at (3,-2.875) {}; % Level 3
    \node[level] at (4,-1.9375) {}; % Level 3
\end{tikzpicture}
\end{center}

