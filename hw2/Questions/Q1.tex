\section*{Q1 [25 Marks]}

Consider the following training data with labels 0 and 1, and three
attributes A, B, and C

\definecolor{Gray}{gray}{0.85} % Define a custom color

\begin{table}[H]
    \centering
    \begin{tabular}{ccccc}
        \hline
        \addlinespace[-0.5ex] % 调整间距 \toprule
        \hline
        id & A & B & C & class \\
        \hline
        1 & 0.62 & yes & yes & 0 \\
        \hline
        2 & 3.84 & no & no & 0 \\
        \hline
        3 & 6.61 & yes & no & 0 \\
        \hline
        4 & 6.87 & yes & no & 0 \\
        \hline
        5 & 7.71 & no & yes & 0 \\
        \hline
        6 & 8.98 & no & yes & 0 \\
        \hline
        7 & 1.77 & yes & no & 0 \\
        \hline
        8 & 2.02 & yes & no & 1 \\
        \hline
        9 & 2.06 & no & yes & 1 \\
        \hline
        10 & 2.66 & no & yes & 1 \\
        \hline
        11 & 3.72 & no & yes & 1 \\
        \hline
        12 & 4.98 & yes & yes & 1 \\
        \hline
        13 & 5.73 & yes & yes & 1 \\
        \hline
        14 & 6.29 & yes & yes & 1 \\
        \hline
        15 & 9.08 & no & no & 1 \\
        \hline
        16 & 9.45 & no & no & 1 \\
        \hline
        \addlinespace[-0.5ex] % 调整间距 \toprule
        \hline
    \end{tabular}
\end{table}

\begin{itemize}
    \item[(a)] (10 marks) Try threshold 2, 5, and 8 for attributes A (that is, use the “A > 2, A < 2”, “A > 5, A < 5”, and “A > 8, A < 8” respectively). Use the Gini score to determine the best one $\theta_a$ among them. Recall:
    \begin{equation*}
        Gini(t) = 1 - \sum_{i = 1}^{c} [p(i|t)]^2  
    \end{equation*}
    \item[(b)] (20 marks) Use $\theta_a$ obtained above, and the Gini score, determine which attributes should firstly be used for developing a decision tree.
\end{itemize}

\subsection*{Solution:}

\paragraph*{(a)} We have: $Info(T) = 1 - (\frac{7}{16})^2 - (\frac{9}{16})^2 = \frac{63}{128} \approx 0.4922$

For threshold 2:

$Info(T_{A<2}) = 1 - \frac{2}{2}^2 - \frac{0}{2}^2 = 0$

$Info(T_{A>2}) = 1 - \frac{5}{14}^2- \frac{9}{14}^2 = \frac{45}{98} \approx 0.4592$

$Info(A_{threshold=2}, T) = \frac{2}{16} \times 0 + \frac{14}{16} \times \frac{45}{98} = \frac{45}{122} \approx 0.4018$

$Gain(A_{threshold=2}, T) = \frac{63}{128} - \frac{45}{122} = \frac{81}{896} \approx 0.0904$

For threshold 5:

$Info(T_{A<5}) = 1 - \frac{3}{8}^2 - \frac{5}{8}^2 = \frac{15}{32} \approx 0.4688$

$Info(T_{A>5}) = 1 - \frac{4}{8}^2- \frac{4}{8}^2 = 0.5$

$Info(A_{threshold=5}, T) = \frac{8}{16} \times \frac{15}{32} + \frac{8}{16} \times \frac{1}{2} = \frac{31}{64} \approx 0.4844$

$Gain(A_{threshold=5}, T) = \frac{63}{128} - \frac{31}{64} = \frac{1}{128} \approx 0.0078$

For threshold 8:

$Info(T_{A<8}) = 1 - \frac{6}{13}^2 - \frac{7}{13}^2 = \frac{84}{169} \approx 0.4970$

$Info(T_{A>8}) = 1 - \frac{1}{3}^2- \frac{2}{3}^2 = \frac{4}{9} \approx 0.4444$

$Info(A_{threshold=8}, T) = \frac{13}{16} \times \frac{84}{169} + \frac{3}{16} \times \frac{4}{9} = \frac{19}{39} \approx 0.4872$

$Gain(A_{threshold=8}, T) = \frac{63}{128} - \frac{45}{122} = \frac{81}{896} \approx 0.0050$

Hence, the best $\theta_a$ is 2.

\paragraph*{(b)}

For attribute B:

$Info(T_{B=yes}) = 1 - \frac{4}{8}^2 - \frac{4}{8}^2 = 0.5$

$Info(T_{B=no}) = 1 - \frac{3}{8}^2- \frac{5}{8}^2 = \frac{15}{32} \approx 0.4688$

$Info(B, T) = \frac{8}{16} \times \frac{1}{2} + \frac{8}{16} \times \frac{15}{32} = \frac{31}{64} \approx 0.4844$

$Gain(B, T) = \frac{63}{128} - \frac{31}{64} = \frac{1}{128} \approx 0.0078$

For attribute C:

$Info(T_{C=yes}) = 1 - \frac{3}{9}^2 - \frac{6}{9}^2 = \frac{4}{9} \approx 0.4444$

$Info(T_{C=no}) = 1 - \frac{4}{7}^2- \frac{3}{7}^2 = \frac{4}{9} \approx 0.4898$

$Info(C, T) = \frac{9}{16} \times \frac{84}{169} + \frac{7}{16} \times \frac{4}{9} = \frac{13}{28} \approx 0.4643$

$Gain(C, T) = \frac{63}{128} - \frac{13}{28} = \frac{25}{896} \approx 0.0279$

Hence, the attribute should firstly be used is attribute A.