\section*{Q2 [30 Marks]}

The table below is a small part of the Acute Inflammations Data Set.

\begin{table}[H]
    \begin{tabular}{ll}
        a1 & Temperature of patient (35C-42C) \\
        a2 & Occurrence of nausea (yes, no) \\
        a3 & Lumbar pain (yes, no) \\
        a4 & Urine pushing (continuous need for urination) (yes, no) \\
        a5 & Micturition pains (yes, no) \\
        a6 & Burning of urethra, itch, swelling of urethra outlet (yes, no) \\
        d1 & Decision: Inflammation of urinary bladder (yes, no) \\
        d2 & Decision: Nephritis of renal pelvis origin (yes, no) \\
    \end{tabular}
\end{table}

Here the attributes a1-a6 are observations, and the decisions d1 and d2 are made by a
medical expert. The purpose of studying this data set is to predict presumptive diagnosis
of two disease of the urinary system, namely, “Inflammation of urinary bladder” and
“Nephritis of renal pelvis origin".

\begin{table}[H]
    \centering
    \begin{tabular}{cccccc|cc}
        \hline
        \addlinespace[-0.5ex] % 调整间距 \toprule
        \hline
        a1 & a2 & a3 & a4 & a5 & a6 & d1 & d2 \\
        \hline
        37.3 & no & yes & no & no & no & no & no \\
        \hline
        37.4 & no & no & yes & no & no & yes & no \\
        \hline
        37.5 & yes & yes & no & no & no & no & no \\
        \hline
        37.6 & no & no & yes & yes & yes & yes & yes \\
        \hline
        37.7 & no & no & yes & no & no & yes & no \\
        \hline
        37.7 & no & no & yes & yes & no & yes & no \\
        \hline
        37.7 & no & no & yes & yes & no & yes & no \\
        \hline
        37.8 & no & yes & no & no & no & no & no \\
        \hline
        37.9 & no & no & yes & yes & yes & yes & no \\
        \hline
        37.9 & no & no & yes & no & no & yes & no \\
        \hline
        38.0 & no & yes & yes & no & yes & no & yes \\
        \hline
        38.0 & no & yes & yes & no & yes & no & yes \\
        \hline
        38.1 & no & yes & yes & no & yes & yes & yes \\
        \hline
        38.3 & no & yes & yes & no & yes & no & yes \\
        \hline
        38.5 & no & yes & yes & no & yes & no & no \\
        \hline
        38.7 & no & yes & yes & no & yes & no & yes \\
        \hline
        38.9 & no & yes & yes & no & yes & yes & yes \\
        \hline
        39.0 & no & yes & yes & no & yes & no & yes \\
        \hline
        39.4 & no & yes & yes & no & yes & no & yes \\
        \hline
        39.5 & no & yes & yes & no & yes & no & yes \\
        \hline
        \addlinespace[-0.5ex] % 调整间距 \bottomrule
        \hline
    \end{tabular}
\end{table}

\begin{itemize}
    \item[(a)] (10 marks) Consider the procedures of building a decision tree with Gini score. If we plan only to use the attributes a3 and a5 to predict the decision d2, which attribute should we use first?
    \item[(b)] (20 marks) Use the naïve Bayes algorithm, the attributes a1 (with the threshold $\theta_1$ = 37.95), a2, and a3 only, to predict the decision d2 for the following data of a new patient. (For simplicity you do NOT need to use the Laplacian correction.)
\end{itemize}

\begin{table}[H]
    \centering
    \begin{tabular}{cccccc|cc}
        \hline
        \addlinespace[-0.5ex] % 调整间距 \toprule
        \hline
        a1 & a2 & a3 & a4 & a5 & a6 & d1 & d2 \\
        \hline
        40.0 & yes & no & no & no & no & ? & ? \\
        \hline
        \addlinespace[-0.5ex] % 调整间距 \bottomrule
        \hline
    \end{tabular}
\end{table}

\subsection*{Solution:}

\paragraph*{(a)} Because we only predict the decision d2, we have: $Info(T)=1-\frac{1}{2}^2-\frac{1}{2}^2=0.5$

For attribute a3, we have:

$Info(T_{yes}) = 1 - \frac{9}{13}^2 - \frac{4}{13}^2 = \frac{72}{169} \approx 0.4260$

$Info(T_{no}) = 1 - \frac{1}{7}^2- \frac{6}{7}^2 = \frac{12}{49} \approx 0.2499$

$Info(a3, T) = \frac{13}{20} \times \frac{72}{169} + \frac{7}{20} \times \frac{12}{49} = \frac{33}{91} \approx 0.3626$

$Gain(a3, T) = 0.5 - 0.3626 = 0.1374$

For attribute a5, we have:

$Info(T_{yes}) = 1 - \frac{1}{4}^2 - \frac{3}{4}^2 = \frac{3}{8} = 0.3750$

$Info(T_{no}) = 1 - \frac{9}{16}^2- \frac{7}{16}^2 = \frac{63}{128} \approx 0.4922$

$Info(a5, T) = \frac{4}{20} \times \frac{3}{8} + \frac{16}{20} \times \frac{63}{128} = \frac{15}{32} \approx 0.4688$

$Gain(a5, T) = 0.5 - 0.4688 = 0.0312$

We can see $Gain(a3, T) > Gain(a5, T)$, so we should use a3 first.

\paragraph*{(b)} Because we have:

For attribute a1:

$P(a1>\theta_1|d2=yes)=\frac{9}{10}=0.9$

$P(a1<\theta_1|d2=yes)=\frac{1}{10}=0.1$

$P(a1>\theta_1|d2=no)=\frac{1}{10}=0.1$

$P(a1<\theta_1|d2=no)=\frac{9}{10}=0.9$

For attribute a2:

$P(a2=yes|d2=yes)=\frac{0}{10}=0$

$P(a2=no|d2=yes)=\frac{10}{10}=1$

$P(a2=yes|d2=no)=\frac{1}{10}=0.1$

$P(a2=no|d2=no)=\frac{9}{10}=0.9$

For attribute a3:

$P(a3=yes|d2=yes)=\frac{9}{10}=0.9$

$P(a3=no|d2=yes)=\frac{1}{10}=0.1$

$P(a3=yes|d2=no)=\frac{4}{10}=0.4$

$P(a3=no|d2=no)=\frac{6}{10}=0.6$

Then we have:

{
    \setlength{\abovedisplayskip}{-10pt}
    \setlength{\belowdisplayskip}{0pt}

    \begin{flalign*}
        P &= P(a1>\theta_1,a2=yes,a3=no) && \\
            &= P(a1>\theta_1) \times P(a2=yes)\times P(a3=no) \\
            &=0.00875
    \end{flalign*}

    \begin{flalign*}
        P_1 &= P(a1>\theta_1,a2=yes,a3=no|d2=yes) && \\
            &= P(a1>\theta_1|d2=yes) \times P(a2=yes|d2=yes)\times P(a3=no|d2=yes) \\
            &=0.9 \times 0 \times 0.1 \\
            &=0
    \end{flalign*}

    \begin{flalign*}
        P_2 &= P(a1>\theta_1,a2=yes,a3=no|d2=no) && \\
            &= P(a1>\theta_1|d2=no) \times P(a2=yes|d2=no)\times P(a3=no|d2=no) \\
            &=0.1 \times 0.1 \times 0.6 \\
            &=0.006
    \end{flalign*}
}

Thus: 

{
    \setlength{\abovedisplayskip}{-10pt}
    \setlength{\belowdisplayskip}{0pt}

    \begin{flalign*}
        &P(d2=yes|a1>\theta_1,a2=yes,a3=no) \\
            &=\frac{P_1 \times P(d2=yes)}{P} && \\
            &=\frac{0 \times 0.5}{0.00875} \\
            &=0
    \end{flalign*}

    \begin{flalign*}
        &P(d2=no|a1>\theta_1,a2=yes,a3=no) \\
            &=\frac{P_2 \times P(d2=no)}{P} && \\
            &=\frac{0.006 \times 0.5}{0.00875} \\
            &=0.3429
    \end{flalign*}
}

Hence, the decision of d2 is no.